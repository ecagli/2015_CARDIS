\begin{abstract}
The two most known projecting linear techniques to effectuate data dimensionality reduction, namely the PCA and the LDA, have raised some issues when introduced in Side-Channel context. The PCA has been proposed in both its supervised and unsupervised version, and opened the question of which principal components are the most suitable to be used for Side-Channel Attacks. The LDA has been valorized for its theoretical  leaning toward the class-distinguishability, but,  because of its computational constraints, has often been set aside. \\
In this paper we present an in depth study of these two methods, and propose a new technique to automatize and ameliorate the selection of principal components, named {\em cumulative ELV selection}. Moreover we present some methods to reduce the constraints to perform the LDA. \\
We equip our study with a comprehensive comparison of the existing and new methods in a real case unified comparison framework, verifying the soundness of the cumulative ELV selection, and the applicability of the methods proposed to find linear discriminant components when LDA is unavailable.
\end{abstract}