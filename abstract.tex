\begin{abstract}
Advanced Side Channel Analyses make use of dimensionality reduction techniques to reduce both the memory and timing complexity of the attack. The  most popular methods to effectuate such a reduction are the Principal Component Analysis (PCA) and the Linear Discriminant Analysis (LDA). They indeed lead to remarkable efficiency gains but their use also raised some issues implied by the Side-Channel context. The PCA provides a set of vectors (the $principal components$) onto which project the data. The open question is which of these principal components are the most suitable for Side-Channel Attacks. The LDA has been valorized for its theoretical  leaning toward the class-distinguishability, but discouraged for its constraining greed of data. In this paper we present an in depth study of these two methods, and we propose a new technique to automatize and ameliorate the selection of principal components, named {\em cumulative Explained Local Variable (ELV) selection}. Moreover we present some extensions of the LDA, available in less constraining situations. We equip our study with a comprehensive comparison of the existing and new methods in real cases. It allows us to verify the soundness of the cumulative ELV selection, and the effectiveness of the methods  proposed to extend the use of the LDA to side channel contexts where the existing approaches are inapplicable.
\end{abstract}