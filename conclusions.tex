\section{Conclusions}\label{sec:conclusions}

In this paper we studied and compared two well-known techniques to construct extractors for side-channel traces, the PCA and the LDA. In our comparison framework we confirmed what expected by theoretical facts: the LDA method is much more adequate than the PCA one, thanks to its class-distinguishing asset. Aware of the PCA issue of choosing suitable components for SCA (observed in two different real case contexts), we proposed a new selection method based on the ELV notion. Thanks to this technique, the class-oriented PCA can achieve in some cases (e.g with our test traces set) performances comparable to or even outperforming those of the LDA. Moreover, it remains a good alternative to LDA when the SSS problem occurs. Finally, among other proposed alternatives to the LDA in presence of the SSS problem, we show that the Direct LDA and the $SW$ Null Space Method are promising, as well.

 %and a deeper analysis is left for the future, together with the inclusion in this comparative study of a third projecting extractor, called {\em Projection Pursuits} CITAZIONE

