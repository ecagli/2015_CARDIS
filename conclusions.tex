\section{Conclusions}\label{sec:conclusions}

In this paper we studied and compared two well-known techniques to construct extractors for side-channel traces, the PCA and the LDA. The LDA method is more adequate than the PCA one, thanks to its class-distinguishing asset, but more expensive and not always available in concrete situations. We deduced from a general consideration about side-channel traces, i.e. the fact that for secured implementations the vulnerable leakages are concentrated into few points, a new methodology  for selecting components, called ELV. We showed that the class-oriented PCA, equipped with the ELV, achieves performances close to those of the LDA, becoming a cheaper and valid alternative to the LDA. Being our core consideration very general in side-channel context, we believe that our results are not case-specific. Finally, among other alternatives to the LDA in presence of SSS problem proposed in Pattern Recognition literature, we showed that the Direct LDA and the $\SW$ Null Space Method are promising, as well.

 %and a deeper analysis is left for the future, together with the inclusion in this comparative study of a third projecting extractor, called {\em Projection Pursuits} CITAZIONE

